%% Generated by Sphinx.
\def\sphinxdocclass{report}
\documentclass[letterpaper,10pt,english]{sphinxmanual}
\ifdefined\pdfpxdimen
   \let\sphinxpxdimen\pdfpxdimen\else\newdimen\sphinxpxdimen
\fi \sphinxpxdimen=.75bp\relax
\ifdefined\pdfimageresolution
    \pdfimageresolution= \numexpr \dimexpr1in\relax/\sphinxpxdimen\relax
\fi
%% let collapsible pdf bookmarks panel have high depth per default
\PassOptionsToPackage{bookmarksdepth=5}{hyperref}

\PassOptionsToPackage{warn}{textcomp}
\usepackage[utf8]{inputenc}
\ifdefined\DeclareUnicodeCharacter
% support both utf8 and utf8x syntaxes
  \ifdefined\DeclareUnicodeCharacterAsOptional
    \def\sphinxDUC#1{\DeclareUnicodeCharacter{"#1}}
  \else
    \let\sphinxDUC\DeclareUnicodeCharacter
  \fi
  \sphinxDUC{00A0}{\nobreakspace}
  \sphinxDUC{2500}{\sphinxunichar{2500}}
  \sphinxDUC{2502}{\sphinxunichar{2502}}
  \sphinxDUC{2514}{\sphinxunichar{2514}}
  \sphinxDUC{251C}{\sphinxunichar{251C}}
  \sphinxDUC{2572}{\textbackslash}
\fi
\usepackage{cmap}
\usepackage[T1]{fontenc}
\usepackage{amsmath,amssymb,amstext}
\usepackage{babel}



\usepackage{tgtermes}
\usepackage{tgheros}
\renewcommand{\ttdefault}{txtt}



\usepackage[Bjarne]{fncychap}
\usepackage{sphinx}

\fvset{fontsize=auto}
\usepackage{geometry}


% Include hyperref last.
\usepackage{hyperref}
% Fix anchor placement for figures with captions.
\usepackage{hypcap}% it must be loaded after hyperref.
% Set up styles of URL: it should be placed after hyperref.
\urlstyle{same}


\usepackage{sphinxmessages}




\title{ERPsim}
\date{May 17, 2022}
\release{0.1}
\author{M2\sphinxhyphen{}G2}
\newcommand{\sphinxlogo}{\vbox{}}
\renewcommand{\releasename}{Release}
\makeindex
\begin{document}

\pagestyle{empty}
\sphinxmaketitle
\pagestyle{plain}
\sphinxtableofcontents
\pagestyle{normal}
\phantomsection\label{\detokenize{index::doc}}


\sphinxAtStartPar
\sphinxstylestrong{ERPsim} est une aide pour le jeu \sphinxstyleemphasis{ERPsim} développé par HEC Montréal.
Ce projet se place dans le cadre d’une observation lors de la Coupe du monde d’ERP. Lors de cette dernière,
plusieurs équipes se sont affrontées sur ERPsim, un jeu sur la gestion d’entreprise. L’une de ces équipes n’était ni plus ni moins qu’un BOT %
\begin{footnote}[1]\sphinxAtStartFootnote
BOT : Agent informatique capable de jouer en autonomie
%
\end{footnote}.

\sphinxAtStartPar
Contre toute attente, c’est le BOT qui a gagné la compétition devant l’ISA et les autres participants. C’est de là qu’est venue l’idée de développer
une Intelligence artificielle capable d’apprendre à jouer mais aussi capable de gagner cette compétition en réagissant aux différents évènements du jeu.

\sphinxAtStartPar
Ce jeu repose sur SAP, un outil indispensable aux entreprises de nos jours. SAP est un ERP
\begin{description}
\item[{ERP (Enterprise Ressource Planning ou Progiciel de Gestion Intégré en français)}] \leavevmode
\sphinxAtStartPar
Dans le principe, un ERP est un logiciel permettant de gérer l’ensemble des processus opérationnels d’une entreprise, que ce soit la gestion des ressources humaines,
la gestion comptable et financière, mais aussi la vente, la distribution, l’approvisionnement ou encore les nouveaux besoins comme le commerce électronique.

\end{description}

\begin{sphinxadmonition}{note}{Note:}
\sphinxAtStartPar
This project is under active development.
\end{sphinxadmonition}

\sphinxAtStartPar
Check out the {\hyperref[\detokenize{Usage::doc}]{\sphinxcrossref{\DUrole{doc}{Usage}}}} section for further information, including how to
{\hyperref[\detokenize{Installation:id1}]{\sphinxcrossref{\DUrole{std,std-ref}{install}}}} the project.


\chapter{Contents}
\label{\detokenize{index:contents}}
\sphinxstepscope


\section{Installation}
\label{\detokenize{Installation:installation}}\label{\detokenize{Installation::doc}}
\begin{sphinxadmonition}{note}{Note:}
\sphinxAtStartPar
Requierements \textendash{} qu’est\sphinxhyphen{}ce qu’il faut installer ? Comment l’installer ? Comment lancer le projet ?
\end{sphinxadmonition}


\subsection{Language}
\label{\detokenize{Installation:language}}\label{\detokenize{Installation:id1}}
\sphinxAtStartPar
This project is developped with Python Language, because it offers …

\sphinxAtStartPar
Pour installer Python, rendez\sphinxhyphen{}vous sur le \sphinxhref{https://www.python.org/downloads/}{site officiel} de Python. Pour information, ce projet a été développé avec Python 3.9.

\sphinxAtStartPar
Vous pouvez retrouver les différents {\hyperref[\detokenize{Installation:packages}]{\sphinxcrossref{\DUrole{std,std-ref}{packages}}}} utilisés ci\sphinxhyphen{}dessous.


\subsection{Packages}
\label{\detokenize{Installation:packages}}\label{\detokenize{Installation:id2}}
\sphinxAtStartPar
To use ERPsim, first install it using pip:

\begin{sphinxVerbatim}[commandchars=\\\{\}]
\PYG{g+gp+gpVirtualEnv}{(.venv)} \PYG{g+gp}{\PYGZdl{} }pip install ...
\end{sphinxVerbatim}

\sphinxAtStartPar
Les différents packages sont :
\begin{itemize}
\item {} 
\sphinxAtStartPar
Django (3.2.11)

\item {} 
\sphinxAtStartPar
python\sphinxhyphen{}dotenv (0.19.2)

\item {} 
\sphinxAtStartPar
python\sphinxhyphen{}dotenv{[}cli{]} (0.19.2)

\item {} 
\sphinxAtStartPar
huey (2.4.3)

\item {} 
\sphinxAtStartPar
redis (4.1.0)

\item {} 
\sphinxAtStartPar
mysql\sphinxhyphen{}connector\sphinxhyphen{}python (8.0.26)

\item {} 
\sphinxAtStartPar
pyodata (1.7.1)

\item {} 
\sphinxAtStartPar
requests (2.23.0)

\end{itemize}

\sphinxAtStartPar
Tous ces packages sont dand le fichier \sphinxstyleemphasis{requierements.txt}
Pour tout installer en une fois, vous pouvez effectuer la commande suivante

\begin{sphinxVerbatim}[commandchars=\\\{\}]
\PYG{g+gp+gpVirtualEnv}{(.venv)} \PYG{g+gp}{\PYGZdl{} }pip install \PYGZhy{}r requirements.txt
\end{sphinxVerbatim}

\sphinxAtStartPar
ou

\begin{sphinxVerbatim}[commandchars=\\\{\}]
\PYG{g+gp+gpVirtualEnv}{(.venv)} \PYG{g+gp}{\PYGZdl{} }pip3 install \PYGZhy{}r requirements.txt
\end{sphinxVerbatim}

\sphinxstepscope


\section{Usage}
\label{\detokenize{Usage:usage}}\label{\detokenize{Usage::doc}}
\begin{sphinxadmonition}{note}{Note:}
\sphinxAtStartPar
Comment utiliser le projet ? Se connecter, utiliser l’application
\end{sphinxadmonition}


\subsection{Getting data}
\label{\detokenize{Usage:getting-data}}\label{\detokenize{Usage:id1}}
\sphinxAtStartPar
To get all the data about the current game
you can use the \sphinxcode{\sphinxupquote{tasks.get\_game\_latest\_data()}} function:
\index{main() (in module manage)@\spxentry{main()}\spxextra{in module manage}}

\begin{fulllineitems}
\phantomsection\label{\detokenize{Usage:manage.main}}
\pysigstartsignatures
\pysiglinewithargsret{\sphinxcode{\sphinxupquote{manage.}}\sphinxbfcode{\sphinxupquote{main}}}{\emph{\DUrole{n}{test}\DUrole{o}{=}\DUrole{default_value}{None}}}{}
\pysigstopsignatures
\sphinxAtStartPar
Run administrative tasks.
\begin{quote}\begin{description}
\item[{Parameters}] \leavevmode
\sphinxAtStartPar
\sphinxstyleliteralstrong{\sphinxupquote{kind}} (\sphinxstyleliteralemphasis{\sphinxupquote{list}}\sphinxstyleliteralemphasis{\sphinxupquote{{[}}}\sphinxstyleliteralemphasis{\sphinxupquote{str}}\sphinxstyleliteralemphasis{\sphinxupquote{{]} or }}\sphinxstyleliteralemphasis{\sphinxupquote{None}}) \textendash{} Optional “kind” of ingredients.

\item[{Raises}] \leavevmode
\sphinxAtStartPar
\sphinxstyleliteralstrong{\sphinxupquote{lumache.InvalidKindError}} \textendash{} If the \sphinxcode{\sphinxupquote{kind}} is invalid.

\item[{Returns}] \leavevmode
\sphinxAtStartPar
The ingredients list.

\item[{Return type}] \leavevmode
\sphinxAtStartPar
list{[}str{]}

\end{description}\end{quote}

\end{fulllineitems}



\subsection{Using data}
\label{\detokenize{Usage:using-data}}
\sphinxAtStartPar
There are x tables, which contains …
This one is for that, and so on …


\subsection{What’s used for visualization}
\label{\detokenize{Usage:what-s-used-for-visualization}}

\subsection{What’s used for helping, prediction}
\label{\detokenize{Usage:what-s-used-for-helping-prediction}}
\sphinxstepscope


\section{Fonctionnement}
\label{\detokenize{Fonctionnement:fonctionnement}}\label{\detokenize{Fonctionnement::doc}}
\begin{sphinxadmonition}{note}{Note:}
\sphinxAtStartPar
Partie plus technique, schéma architectural du projet, différentes technologies mises en jeux et leurs intercommunications ?
\end{sphinxadmonition}
\phantomsection\label{\detokenize{Fonctionnement:id1}}
\sphinxstepscope


\section{Documentation du Code}
\label{\detokenize{CodeSource:documentation-du-code}}\label{\detokenize{CodeSource::doc}}
\begin{sphinxadmonition}{note}{Note:}
\sphinxAtStartPar
Expliquer tout le code dans le détail, comment ça fonctionne ?
\end{sphinxadmonition}
\phantomsection\label{\detokenize{CodeSource:codesource}}\subsubsection*{Notes}



\renewcommand{\indexname}{Index}
\printindex
\end{document}